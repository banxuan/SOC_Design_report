\section{NAND logic gate}
In this exercise the design of the NAND gate was expected to have minimum area while to have high fanout output(fanout =8). In order to find the delay time between input and output of the NAND gate, the Elmore Delay model~\cite{elmore1948transient} was used in the design.\\
\subsection{Design and Optimization}
As shown in the Figure~\ref{fig:nand}, NAND gate is constructed by two parallel connected PMOS transistors and two serial connected NMOS transistors. 

\begin{figure*}[ht]
\centering
\includegraphics[width = 0.7\textwidth]{Figures/nand}
\caption{NAND gate schematic}
\label {fig:nand}
\end{figure*}

Although transistors have complicate current-voltage behaviour, turned on transistors can be assumed as resistors, a chain of transistors can be represented as an RC ladder, as shown in Figure~\ref{fig:elmore}. Therefore the Elmore delay model can estimate the delay of this RC ladder in terms of the path resistance and capacitance of a node on the ladder and the supply:

\begin{align}
	{t\textsubscript{pd} = \sum_{\substack{i}} R\substack{n-i}C\substack{i}}
\end{align}

\begin{figure}[H]
		\centering
		%\includegraphics[width = 0.9\textwidth]{Figures/Overview_of_presentations}
		\includegraphics[width = 0.45\textwidth]{Figures/elmoredelaymodel}		
		\caption{Simple Elmore delay model}
		\label {fig:elmore}
\end{figure}
Therefore it is possible to use this method as the reference to calculate the width of logic gate, since the propagation delay is dependent to it and the capacitance of the load. \\
Typically in the logic gate, for parallel connected  transistors, the total resistance is lower when they are all on. In many gates, the worst-case delay is usually because only on the parallel transistor is on~\cite{[2]}. So when applying the Elmore delay model in the NAND design, one of input is connected to vdd while the other one is connected in the path, as shown in Figure~\ref{fig:nandtest}.
\begin{figure}[H]
		\centering
		%\includegraphics[width = 0.9\textwidth]{Figures/Overview_of_presentations}
		\includegraphics[width = 0.5\textwidth]{Figures/nandtest}		
		\caption{NAND Elmore delay model}
		\label {fig:nandtest}
\end{figure}

While deceasing the width of NMOS and increasing the width of PMOS, the propagation delays, which is the maximum time from input to output crossing 50\%, of the rising edge and falling edge are also changing. When the certain ratio between the width of PMOS and NMOS is reached, the propagation delay between rising edge and falling edge of logic gate will become similar. In another word, the signal slope at the output of logic gate will become quite identical to the input, only with an amount of delay.\\
In the design, the ratio between PMOS and NMOS is 2.3, as found in the Figure~\ref{fig:NANDratio}.
\begin{figure}[hf]
		\centering
		%\includegraphics[width = 0.9\textwidth]{Figures/Overview_of_presentations}
		\includegraphics[width = 0.5\textwidth]{Figures/NANDratio}		
		\caption{Ratio of width of PMOS and NMOS in NAND gate}
		\label {fig:NANDratio}
\end{figure}

If a gate can drive n copies of itself, then it is said to have a fanout or electrical effort of n. When multiple gates are chain connected together to form a multi-stage logic network, Since the ratio between size of PMOS and NMOS is confirmed, the next procedure is determining the actual size of transistors in the NAND gate.
